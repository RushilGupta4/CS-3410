\documentclass[a4paper]{article}
\setlength{\topmargin}{-1.0in}
\setlength{\oddsidemargin}{-0.2in}
\setlength{\evensidemargin}{0in}
\setlength{\textheight}{10.5in}
\setlength{\textwidth}{6.5in}
\usepackage{enumitem}
\usepackage{amsmath}
\usepackage{hyperref}
\usepackage{amssymb}
\usepackage[dvipsnames]{xcolor}
\usepackage{mathpartir}
\usepackage{algorithm}
\usepackage{algpseudocode}
\usepackage{csquotes}
\usepackage{bbm}

\hbadness=10000

\hypersetup{
    colorlinks=true,
    linkcolor=blue,
    filecolor=magenta,      
    urlcolor=cyan,
    pdftitle={Assignment 2B},
    pdfpagemode=FullScreen,
}
\def\endproofmark{$\Box$}
\newenvironment{proof}{\par{\bf Proof}:}{\endproofmark\smallskip}

\begin{document}
\begin{center}
{\large \bf \color{red} Department of Computer Science} \\
{\large \bf \color{red} Ashoka University} \\

\vspace{0.1in}

{\large \bf \color{blue} Introduction to Machine Learning}

\vspace{0.05in}

{ \bf \color{YellowOrange} Assignment 2B}
\end{center}
\medskip

{\textbf{Collaborators:} None} \hfill {\textbf{Name: Rushil Gupta} }

\bigskip
\hrule

\section*{Question 1}
Done on the jupyter notebook.

\section*{Question 2}
Done on the jupyter notebook.

\section*{Question 3}
Done on the jupyter notebook.

\newpage
\section*{Question 4}
\begin{enumerate}
    \item \textbf{$h(x) = \mathbbm{1}{(a < x)}, a \in \mathbb{R}$}\\

    Claim: The VC dimension of this hypothesis class is 1.
    
    To shatter 1 point: Given a single point $x_1$, we can achieve both labelings by setting $a$ appropriately:
    \begin{itemize}
        \item For $(+)$: Choose $a < x_1$
        \item For $(-)$: Choose $a > x_1$
    \end{itemize}
    
    Impossible to shatter 2 points: For points $x_1 < x_2$, the labeling $(+,-)$ cannot be achieved with any choice of threshold $a$, as any $a$ that makes $x_1$ positive (i.e., $a < x_1$) would also make $x_2$ positive since $x_1 < x_2$. This threshold function can only create a single decision boundary, resulting in all points to the right being positive and all points to the left being negative.\\

    \item \textbf{$h(x) = \mathbbm{1}{(a < x < b)}, a, b \in \mathbb{R}$}

    Claim: The VC dimension of this hypothesis class is 2.
    
    To shatter 2 points: Given points $x_1 < x_2$, all four labelings can be achieved by setting $a$ and $b$ appropriately:
    \begin{itemize}
        \item For $(+,+)$: Choose $a < x_1$ and $b > x_2$
        \item For $(-,-)$: Choose $a > x_2$ or $b < x_1$
        \item For $(-,+)$: Choose $a < x_2 < b$ with $a > x_1$
        \item For $(+,-)$: Choose $a < x_1 < b < x_2$
    \end{itemize}
    
    Cannot shatter 3 points: For points $x_1 < x_2 < x_3$, the labeling $(+,-,+)$ is impossible with this hypothesis class. If $x_1$ and $x_3$ are within the interval (positive), then $x_2$ must also be within the interval (positive) since the hypothesis represents a single connected interval. This shows that any arrangement of 3 points on a line cannot be shattered by this hypothesis class.\\

    \item \textbf{$h(x) = \mathbbm{1}{(a \sin x > 0)}, a \in \mathbb{R}$}
    
    Claim: The VC dimension of this hypothesis class is 1.
    
    To shatter 1 point: For any single point $x_1$ with $\sin x_1 \neq 0$, both labelings can be achieved:
    \begin{itemize}
        \item For $(+)$: Choose $a$ with the same sign as $\sin x_1$, making $a \sin x_1 > 0$
        \item For $(-)$: Choose $a = 0$ or $a$ with the opposite sign as $\sin x_1$, making $a \sin x_1 \leq 0$
    \end{itemize}
    
    Impossible to shatter 2 points: For any two points $x_1, x_2$:
    \begin{itemize}
        \item If $\sin x_1$ and $\sin x_2$ have opposite signs, the labeling $(+,+)$ is impossible.
        \item If they have the same sign, the labelings $(+,-)$ and $(-,+)$ are impossible.
    \end{itemize}
    
    This is because the parameter $a$ can only control the sign of the hypothesis, giving us at most three distinct functions: positive when $\sin x > 0$ (when $a > 0$), positive when $\sin x < 0$ (when $a < 0$), or identically 0 (when $a = 0$). Therefore, the VC dimension is 1.\\

    \item \textbf{$h(x) = \mathbbm{1}{(\sin(x + a) > 0)}, a \in \mathbb{R}$}
    
    Claim: The VC dimension of this hypothesis class is 2.
    
    To shatter 2 points: For any two points $x_1, x_2$ whose distance is less than $\pi$ (e.g., $x_2 = x_1 + \frac{\pi}{2}$), all four labelings can be achieved by selecting an appropriate shift $a$:
    \begin{itemize}
        \item For $(+,+)$: Choose $a$ so that both $x_1 + a$ and $x_2 + a$ lie within a region where $\sin$ is positive
        \item For $(-,-)$: Choose $a$ so that both $x_1 + a$ and $x_2 + a$ lie within a region where $\sin$ is negative
        \item For $(+,-)$: Choose $a$ so that $x_1 + a$ is in a positive region and $x_2 + a$ is in a negative region
        \item For $(-,+)$: Choose $a$ so that $x_1 + a$ is in a negative region and $x_2 + a$ is in a positive region
    \end{itemize}
    
    Cannot shatter 3 points: For any three points $x_1 < x_2 < x_3$, the labeling $(+,-,+)$ is impossible. If $\sin(x_1 + a) > 0$ and $\sin(x_3 + a) > 0$, then $x_1 + a$ and $x_3 + a$ must lie in the same positive interval of length $\pi$. Since this interval is contiguous, the entire segment between them, including $x_2 + a$, must also lie in this positive region, forcing $h_a(x_2) = 1$. This contradiction shows that no set of three points can be shattered, so the VC dimension is 2.
\end{enumerate}



\newpage
\section*{Question 5}
\section*{Question 5.1}
\begin{enumerate}
    \item \textbf{(a)} Can be shattered.
    \item \textbf{(b)} Cannot be shattered.
    \item \textbf{(c)} Can be shattered.
\end{enumerate}

\section*{Question 5.2}
Since the hypothesis class could shatter a set of 4 points, it has a VC dim of at least 4.

\end{document}